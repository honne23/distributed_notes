% load lecture note class
\documentclass{easyclass}
\begin{document}
\begin{titlepage}
    \university{University of Surrey}
    \courseid{COM3026}
    \title{Distributed Computing Lecture Notes}
    \author{Adrian Coutsoftides}
    \version{9102 Spring}
    \instructor{Instructors: Dr.~Ins \textsc{Tructor1}\par
    	Dr.~Ins \textsc{Tructor2}, Dr.~Ins \textsc{Tructor3}\par}
    \maketitle
\end{titlepage}

\tableofcontents

\chapter{Moore's law}

\chapter{Network partition}

\chapter{Marshaling}

\chapter{Berkely Algorithm}

\chapter{Cristian's algorithm}

\chapter{Lamport algorithm}

\chapter{Mutual Exclusion Algorithm}

\chapter{Leader Election Algorithms}

\chapter{Deadlock}

\chapter{Salt}

\chapter{Symmetric encryption}

\chapter{TCP \& UDP}
\section{TCP}
\section{UDP}

\chapter{Kerberos}
\begin{itemize}
    \item Developed by MIT
    \item Trusted third party
    \item Symmetric crypto
    \item Passwords not sent in clear text
    \begin{itemize}
        \item Assumes only network is compromised
    \end{itemize}
\end{itemize}
\chapter{Challenge-handshake}

\chapter{Diffie-Hellman (L09)}
Secure key distribution is the biggest problem with symmetric cryptography.

\begin{theo}[Diffie-Hellman Key Exchange]{}
    \begin{itemize}
        \item First algorithm to use public / private keys
        \item \emph{Not} public key encryption
        \item Uses a one-way function 
        \begin{itemize}
            \item Based on difficulty of computing discrete 
            logarithms in a finite field compared with ease of calculating exponentiation
        \end{itemize}
    \end{itemize}
    $\rightarrow$ Allows us to negotiate a secret \textbf{common key} without fear of evesdroppers\\
    $\rightarrow$ All arithmetic operations are performed in a field of integers modulo some large number
    \begin{itemize}
        \item Both parties agree on a large prime $p$ and a number $\alpha < p$
        \item Each party generates a public / private key pair
        \begin{itemize}
            \item Private key for user $i$: $X_i$
            \item Public key for user $i$: $Y_i = \alpha^{X_i} \mod p$
        \end{itemize}
    \end{itemize}
\end{theo}

\begin{lem}[Diffie Hellman exponential key exchange]{}
    \begin{minipage}[t]{0.45\linewidth}
        \centering
        \begin{itemize}
            \item Alice has a secret key  $X_A$
            \item Alice has a public key $Y_A$
            \item Alice computes $K = Y_B^{X_A} \mod p$
        \end{itemize}
        \end{minipage}
        \hspace{0.5cm}
        \begin{minipage}[t]{0.45\linewidth}
            \begin{itemize}
                \item Bob has a secret key $X_B$
                \item Bob has a public key  $Y_B$
                \item Bob computes $K' = Y_A^{X_B} \mod p$
            \end{itemize}
        \centering
    \end{minipage}

    \begin{definition}{}
        Given that $K = K'$, this means $K$ is a \textbf{common key} known only to Bob and Alice 
    \end{definition}
\end{lem}
\newpage
\section{RSA Public Key Cryptography}
    \begin{itemize}
        \item Each user generates 2 keys
        \begin{itemize}
            \item \emph{Private key} (kept secret)
            \item \emph{Public key} (can be shared with anyone)
        \end{itemize}
    \end{itemize}
    $\rightarrow$ Based on the difficulty of factoring large numbers
    \begin{theo}[Public-key algorithm]{}
        \begin{itemize}
            \item Two related keys:
            \begin{itemize}
                \item $C = E_{K1}(P)$,  $P = D_{K2}(C)$
                \item $C' = E_{K2}(P)$,  $P = D_{K1}(C')$
                \item $K_1$ is public, $K_2$ is private
            \end{itemize}
            \item Examples:
            \begin{itemize}
                \item RSA \& Elliptic Curve Algorithms
                \item DSS (digital signature standard)
            \end{itemize}
            \item Key length:
            \begin{itemize}
                \item Not every number is a valid key (unlike symmetric crypto)
                \item 3072-bit RSA = 256-bit elliptic curve = 128-bit symmetric cipher
                \item 15360-bit RSA = 521-bit elliptic curve = 256-bit symmetric cipher
            \end{itemize}
        \end{itemize}

        Different keys for encrypting and decrypting
        \begin{itemize}
            \item No need to worry about key distribution
            \item Share public keys
            \item Keep private keys secret
        \end{itemize}
        $\rightarrow$ Alice encrypts her message with Bob's public key, and Bob 
        decrypts it with his private key
    \end{theo}

    \section{Session keys}
    \begin{definition}{}
        Randomly generated key for one communication session
    \end{definition}

    \begin{itemize}
        \item Use public key to send the session key
        \item Use symmetric algorithm to encrypt data with the session key
    \end{itemize}

    \begin{prf}[Security]{}
        \begin{itemize}
            \item Public key algorithms are almost never used to encrypt messages
            \begin{itemize}
                \item Much slower, vulnerable to \emph{chosen-plaintext} attacks
                \item RSA-2048 approximately 55x slower to encrypt and 2,000x slower to  decrypt than AES-256
            \end{itemize}
        \end{itemize}
    \end{prf}

    \begin{theo}[Communication using hybrid crypto-system]{}
        \begin{itemize}
            \item Encrypt message using symmetric algorithm and key $K$
            \item Decrypt message using symmetric algorithm and key $K$
        \end{itemize}
        
    \end{theo}

    \section{Hash functions}
    \begin{itemize}
        \item Easy to compute in one direction
        \item Difficult to comput the other way round
    \end{itemize}
    $\rightarrow$ discrete logs or prime factoring \\
    $\rightarrow$ difficult implies no known shortcuts, requires exhaustive search

    \begin{lem}[Digital Signatures]{}
        \begin{itemize}
            \item Validate
            \begin{enumerate}
                \item The creator (signer) of the content
                \item The content has not been modified since it was signed
                \item Content itself does not have to be encrypted
            \end{enumerate}
        \end{itemize}
        \textbf{Encrypting a message with a private key is the same as signing it} \\
        But this is not ideal:
        \begin{itemize}
            \item We don't want to permute or hide the content
            \item We just want Bob to verify that the content came from Alice
            \item Public key cryptography is much slower than symmetric encryption
            \item What if the input was multiple \texttt{GB}
        \end{itemize}
    \end{lem}

    \begin{theo}[Hash Functions]{}
        \begin{definition}{}
            Cryptographic hash function, also known as a digest
        \end{definition}
        \begin{itemize}
            \item Input: arbitary bytes
            \item Output: fixed-length bit-string
        \end{itemize}

        \begin{itemize}
            \item Properties
            \begin{itemize}
                \item One-way function
                \begin{itemize}
                    \item Given a hash $H = hash(M)$ it should be difficult to compute $M$ given $H$
                \end{itemize}
                \item Collision resistant
                \begin{itemize}
                    \item Given a hash $H = hash(M)$ it should be difficult to find $M'$ s.t $H = hash(M')$
                    \item For a hash of length $L$, a perfect hash would take $2^{L/2}$ attempts
                \end{itemize}
                \item Efficient
                \begin{itemize}
                    \item Computing a hash function should be computationally efficient
                \end{itemize}
            \end{itemize}
        \end{itemize}
        
    \end{theo}

    \begin{prf}[Popular hash functions]{}
        \begin{itemize}
            \item SHA-2
            \item SHA-3
            \begin{itemize}
                \item NIST standard as of 2015
            \end{itemize}
            \item MD5
            \begin{itemize}
                \item 128 bits
            \end{itemize}
            \item Hash functions derived from cyphers
            \begin{itemize}
                \item Blowfish (used for password hashing in openBSD)
                \item 3DES - used for old linux password hashes
            \end{itemize}
        \end{itemize}
    \end{prf}
\newpage
    \begin{theo}[Digital signatures with hash functions]{}
        \begin{itemize}
            \item Sender
            \begin{itemize}
                \item Create hash of message
                \item Encrypt hash with \textbf{private-key} \& send it with message
            \end{itemize}
            \item Recipient 
            \begin{itemize}
                \item Decrypts the encrypted hash using your public key
                \item Computes hash of the received message
                \item Compare the descrypted hash with the message hash
                \item If they're the same the message has not been modified
            \end{itemize}
        \end{itemize}
    \end{theo}

    \begin{prf}[Message Auth Codes vs. Signatures]{}
        \begin{minipage}[t]{0.45\linewidth}
            \centering
            \begin{itemize}
                \item \textbf{Message authentication code (MAC)}
                \item Hash of message encrypted with a symmetric key
            \end{itemize}
            \end{minipage}
            \hspace{0.5cm}
            \begin{minipage}[t]{0.45\linewidth}
                \begin{itemize}
                    \item \textbf{Digital signature}
                    \item Hash of the message encrypted with the owner's private key
                    \begin{itemize}
                        \item Alice encrypts the hash with her private key
                        \item Bob validates it by decrypting it with her public key \& comparing
                        with $hash(M)$
                    \end{itemize}
                    \item Provides non-repudiation
                \end{itemize}
            \centering
        \end{minipage}
        \begin{definition}{}
            Non-Repudiation means the recipient cannot change the encrypted hash
        \end{definition}
    \end{prf}

    \section{Digital Signatures with public key crypto}

    \begin{itemize}
        \item Alice generates hash of message
        \item Alice encrypts has with her \textbf{private key}, this is her signature
        \item Alice sends bob the message and the encrypted hash
        \item Bob decypts the hash using Alice's public key
        \item Bob computes the hash of the message sent by Alice
        \item \textbf{If the hashes match, the signature must be valid}
    \end{itemize}

    \subsection{Multiple signers}
    \begin{itemize}
        \item Charles
        \begin{itemize}
            \item Generates hash of message $H(P)$
            \item Decrypts Alice's signature with her public key
            \begin{itemize}
                \item Validates the signature $D_A(S) = H(P)$
            \end{itemize}
            \item Decypts Bob's signature with his public key
            \begin{itemize}
                \item Validates the signature $D_B(S) = H(P)$
            \end{itemize}
        \end{itemize}
    \end{itemize}

    \subsection{Covert \& Authenticated Messaging}
    Combine encryption with digital signatures

    \begin{theo}[Covert Authenticated Messages]{}
        Use a \textbf{session key}:
        \begin{itemize}
            \item Pick a random key, $K$ to encrypt the message with a symmetric algorithm
            \item Encrypt $K$ with the public key of each recipient
            \item For \emph{signing}, \textbf{encrypt the hash} of each message with
            the sender's private key
        \end{itemize}
    \end{theo}
\chapter{2-phase / 3-phase commit protocol}

\chapter{Map-Reduce}

\chapter{PBFT (L11)}
\begin{theo}[Historical Context]{}
    \begin{itemize}
        \item $N$ generals, $f$ of them are traitors
        \item Exchange plans by messengers
        \item \begin{itemize}
            \item Unreliable %
        \end{itemize}
        \item All loyal generals agree on same plan of action
        \item Chosen plan must be poposed by loyal general
    \end{itemize}
\end{theo}


\begin{lem}[Computer Science Setting]{}
    \begin{itemize}
        \item A general $\Leftrightarrow$ A program / process / replica
        \item \begin{itemize}
            \item Replicas communicate
            \item Traitors $\Leftrightarrow$ Failed replicas
        \end{itemize}
        \item Byzantine army $\Leftrightarrow$ Deterministic replicated service
        \item \begin{itemize}
            \item Service has states and some operations
            \item Service should cope with failures
            \item \begin{itemize}
                \item State should be consistent across all replicas
            \end{itemize}
        \end{itemize}
    \end{itemize}
\end{lem}
\begin{prf}[Applications]{}
    \begin{itemize}
        \item Replicted file systems
        \item Backup
        \item Distributed Servers
        \item Shared ledgers between banks
    \end{itemize}
\end{prf}

\section{Byzantine fault tolerance}
\begin{itemize}
    \item Distributed computing with faulty replicas
     \begin{itemize}
        \item $N$ replicas
        \item $f$ of them faulty
        \item Replicas initially start in the same state
    \end{itemize}
    \item For a given request:
    \begin{itemize}
        \item Gurantee that all non faulty replicas agree on the next state
        \item Provide system consistency even when some replicas may be inconsistent
    \end{itemize}
\end{itemize}

\begin{prf}[Properties]{}
    \begin{itemize}
        \item Safety
        \begin{itemize}
            \item \emph{Agreement}: All non-faulty relpicals agree on the same state
            \item \emph{Validity}: The chosen state is valid
        \end{itemize}
        \item Liveness
         \begin{itemize}
            \item Some state is eventually agreed upon
            \item If a state is chosen all replicas eventually arrive at the same state
        \end{itemize}
    \end{itemize}
\end{prf}


\begin{prf}[Paxos vs BFT]{}
    \begin{minipage}[t]{0.45\linewidth}
        \centering
        \begin{itemize}
            \item [\textbf{Paxos}]
            \item Async Network
            \item Tolerated crash failure
            \item Guranteed safety but not Liveness
            \item Protocol may not terminate
            \item Terminate if the network is synchronous eventually
            \item \textbf{Require at least $3f+1$ replicas to tolerate $f$ faulty replicas}
        \end{itemize}
        \end{minipage}
        \hspace{0.5cm}
        \begin{minipage}[t]{0.45\linewidth}
            \begin{itemize}
                \item [\textbf{BFT}]
                \item Better performance
                \item Model in BFT is practical
                \item Adoption in inducstry
            \end{itemize}
        \centering
    \end{minipage}
\end{prf}
\newpage
\begin{theo}[Byzantine System Model]{}
    \begin{itemize}
        \item Asynchronous distributed system
                \begin{itemize}
                    \item Delay, duplicate or deliver messages out of order
                \end{itemize}
                \item Byzantine failure model
                 \begin{itemize}
                    \item Faulty replicas may behave arbitarily
                \end{itemize}
                \item Preventing spoofing, relays and corrupted messages
                \begin{itemize}
                    \item Public-key signature: one cannot impersonate the other
                    \item Message authentication code, collision resistant hash: one cannot tamper
                    the other's messages
                \end{itemize}
    \end{itemize}
\end{theo}

\begin{theo}[Advesary Model]{}
    \begin{itemize}
        \item Can coordinate faulty replicas
        \item Delay communications but not indefinetly
        \item Cannot subvert cryptographic techniques employed
    \end{itemize}
\end{theo}

\begin{theo}[Service Properties]{}
    \begin{itemize}
        \item Safety
        \item Liveness
        \item Optimal resilliency 
        \begin{itemize}
            \item To tolerate $f$ fulty replicas the system requires $n = 3f + 1$ replicas
            \item Can proceed after communicating with $n-f \backsim 2f+1$ replicas
            \begin{itemize}
                \item If none of those $2f+1$ replicas is faulty, good
                \item Even if $f$ are faulty, the majority vote ($f+1$) ensures safety
            \end{itemize}
        \end{itemize}
    \end{itemize}
\end{theo}
\newpage
\begin{prf}[Algorithm]{}
\begin{itemize}
    \item The set of relicas is $R$ where $\vert R \vert = 3f+1$
    \item Each replica identified by UID $0, \hdots 3f$
    \item Each replica is deterministic and starts at the same initial state
    \item \textbf{A view is a configuration of replicas}
    \begin{itemize}
        \item Replica $p = v \mod \vert R \vert$ is the primary of view $v$
        \item All other replicas are backups
    \end{itemize}
\end{itemize}

\begin{enumerate}
    \item Client sents request to primary
    \item Primary validates request and initiates 3-phase protocol to ensure consensus:
    \[\text{Pre-Prepare} \rightarrow \text{Prepare} \rightarrow \text{Commit}\]
    \item Replicas execute request and send it directly back to the client
    \item Client accepts result after receiving $f+1$ identical replies
\end{enumerate}
\end{prf}

\begin{theo}[Three phase protocol goals]{}
    Ensure safety and liveliness despite asynchronous nature
    \begin{itemize}
        \item Establish total order of execution requests (\emph{Pre-Prepare + Prepare})
        \item Ensure requests are ordered consistently across views (\emph{Commit})
    \end{itemize}
\end{theo}

\begin{lem}[Three phases]{}
    \begin{itemize}
        \item Pre-Prepare
        \begin{itemize}
            \item Achknowledge a unique sequence number for the request
        \end{itemize}
        \item Prepare
        \begin{itemize}
            \item The replicas agree on this sequence number
        \end{itemize}
        \item Commit
        \begin{itemize}
            \item Establish total order across views
        \end{itemize}
    \end{itemize}
    \[\text{Request} \rightarrow \text{Pre-Prepare} \rightarrow \text{Prepare} \rightarrow \text{Commit} \rightarrow \text{Reply}\]
\end{lem}
\section{Three phase protocol}
\begin{theo}[Definitions]{}
    \begin{itemize}
        \item Request message: $m$
        \item Sequence number: $n$
        \item Signature: \textyen
        \item View: $v$
        \item Primary replica: $p$
        \item Digest of message $D(m) \rightarrow d$
    \end{itemize}
\end{theo}

\begin{prf}[Pre-Prepare]{}
    Purpose: acknowledge a unique sequence number for the request
    \begin{itemize}
        \item \textbf{Send} 
        \begin{itemize}
            \item The primary assigns the request a sequence number and broadcasts
            this to all replicas
        \end{itemize}
        \item A backup will \textbf{Accept} a message \emph{iff}:
        \begin{itemize}
            \item $d,v,n, \text{\textyen}$ are valid
            \item ($v,n$) has not been processed before for another digest $d$
        \end{itemize}
    \end{itemize}
\end{prf}

\begin{prf}[Prepare]{}
    Purpose: the replicas agree on this sequence number, \\
    $\rightarrow$ this is after backup $i$ accepts \texttt{<Pre-Prepare>} message
    \begin{itemize}
        \item \textbf{Send}
        \begin{itemize}
            \item multicast a \texttt{<Prepare>} message acknowledging $n,d,i,v$
        \end{itemize}
        \item A replica \textbf{Accepts} the message \emph{iff}:
        \begin{itemize}
            \item $d,v,,n,\text{\textyen}$ are valid
        \end{itemize}
    \end{itemize}
\end{prf}

\begin{prf}[Prepared]{}
    Predicate $(m,v,n,i) = \text{\texttt{True}}$ \emph{iff} for replica $i$:
    \begin{itemize}
        \item \texttt{<Pre-Prepare>} for $m$ has been received
        \item $2f+1$ (including self) is distinct and 
        corresponding \texttt{<Prepare>} messages received.
    \end{itemize}
\end{prf}

\begin{prf}[Commit]{}
    Purpose: establish order across views, \\
    Once $(m,v,n,i) = \text{\texttt{True}}$ is prepared for replica $i$:
    \begin{itemize}
        \item \textbf{Send}
        \begin{itemize}
            \item multicast \texttt{<Commit>} message to all replicas
        \end{itemize}
        \item A replica \textbf{Accepts} the message \emph{iff}:
        \begin{itemize}
            \item $d,v,,n,\text{\textyen}$ are valid    
        \end{itemize}
    \end{itemize}
\end{prf}

\begin{lem}[Commited]{}
    Predicate $(m,v,n,i) = \text{\texttt{True}}$ commited \emph{iff} for replica $i$:
    \begin{itemize}
        \item \texttt{prepared}$(m,v,n,i) = \text{\texttt{True}}$
        \item $2f+1$ (including self) distinct and corresponding valid \texttt{<Commit>} message received
    \end{itemize}
\end{lem}

\begin{theo}[Executing Requests]{}
    Replica $i$ executes request \emph{iff}:
    \begin{itemize}
        \item \texttt{commited}$(m,v,n,i) = \text{\texttt{True}}$
        \item All requests with slower $n$ are already executed
    \end{itemize}
    Once executed, the replicas will directly sent \texttt{<Reply>} to the client
\end{theo}

Everything proceeds as normal if the primary is good, but when
the primary is faulty this can cause issues.
\section{View Change}
\begin{lem}[View Change]{}
    If the replica receives requests \{1,2,4,5\}, it waits to receive
    request 3 before processing 4 \& 5. \\
    Whenever a lot of non-faulty replicas detect that the primary is faulty,
    they together begin the \texttt{view\_change} operation.
    \begin{itemize}
        \item If they are stuck, they suspect the primary is faulty
        \item This fault is detected using timeouts
        \item This depends in part on the synchrony assumption
        \item They will then change the view
        \begin{itemize}
            \item $p \leftarrow (p+1) \mod \vert R \vert$
        \end{itemize}
    \end{itemize}
\end{lem}

\begin{prf}[Initiating the View Change]{}
    \begin{itemize}
        \item Every replica that  wants to begin a view change sends a
        \texttt{<View-Change>} message to \texttt{Everyone}.
        \begin{itemize}
            \item Includes current state so that all replicas will know which
            requests haven't been commited yet (due to faulty primary)
            \item List of requests that were prepared
        \end{itemize}
        \item When the new primary receives $2f+1$ \texttt{<View-Change>} messages,
        it will begin the view change
    \end{itemize}
\end{prf}
\newpage
\begin{prf}[View-Change \& Correctness]{}
    \begin{itemize}
        \item New primary gathers information about which requests need committing 
        \begin{itemize}
            \item This information is included in the \texttt{<View-Change>} message 
            \item All replicas can also compute this sicne they also receive the 
            \texttt{<View-Change>} message.
            \begin{itemize}
                \item Will avoid faulty new primary making the state inconsistent
            \end{itemize}
        \end{itemize}
        \item New primary sends \texttt{<New-View>} to all replicas
        \item All replicas perform 3 phases on request again
    \end{itemize}
\end{prf}

\begin{lem}[View Change fixing missing request]{}
    \textbf{Sequence numbers with missing requests are replaced with a } \emph{"no-op / pass"} \textbf{operation (null operation).}
\end{lem}

\begin{theo}[State Recomputation]{}
    \begin{itemize}
        \item New primary needs to compute which requests need to be commited again
        \item Redoing all requests is expensive
        \item Use checkpoints to speed up progress
        \begin{itemize}
            \item Every 100 steps, all replicas save their current state into a checkpoint
            \item Replicas should agree on the checkpoint as well
        \end{itemize}
    \end{itemize}
\end{theo}
Other types of problems include:
\begin{itemize}
    \item New primary also faulty
    \begin{itemize}
        \item Use another time-out in the view change
        \begin{itemize}
            \item When the timeout expires another replica will be chosen as the primary
            \item Since there are at most $f$ faulty replicas, \textbf{the primary can be
            consecutively faulty for at most $f$ times}
        \end{itemize}
    \end{itemize}
    \item Primary might pick disproportionately large sequence number $\backsim 1e^{10}$
    \begin{itemize}
        \item The sequence number must lie within a certain interval
        \item This interval is updated periodically
    \end{itemize}
\end{itemize}
\section{Client full protocol}
The client may send a request to the primary, but the primary
doesnt forward this request to the replicas
\begin{theo}[Client Full Protocol]{}
    \begin{itemize}
        \item Client sends a request to the primary  that they knew
        \begin{itemize}
            \item The primary may already change, this will be handled
        \end{itemize}
        \item If they do not receive a reply within a period of time, it broadcasts the
        request to all replicas
    \end{itemize}
\end{theo}

\begin{theo}[Replica Protocol]{}
    \begin{itemize}
        \item If a replica receives a request from the client but not from the primary,
        they will forward this request to the primary.
        \item If they then do not receive a reply from the primary, they begin
        the \texttt{view\_change} operation.
    \end{itemize}
\end{theo}

\section{Correctness of View Change}
A key proof illustrates why the \texttt{view\_change} operation 
preserves \textbf{safety}.

\begin{theo}[Correctness of View Change]{}
    \begin{definition}
        If at any moment the replica has commited a request, this request
        will \emph{always} be commited in the \texttt{view\_change}
    \end{definition}
    \begin{itemize}
        \item A request is re-commited if they are included in at least 
        one of the \texttt{<View-Change>} messages
        \item A commited request implies there is at least $f+1$ non-faulty replicas
        that \emph{prepared} it.
        \begin{proof}
            ...
            \begin{itemize}
                \item $\exists \; 2f+1$ \texttt{<View-Change>} messages
                \item $\forall m, \; \vert \: \text{\texttt{prepared}}(m) \: \vert \geq f+1$
                \item If $\vert R \vert = 3f + 1$, at least one faulty replica must have prepared
                $m$ and sent the \texttt{<View-Change>} message.
            \end{itemize}
        \end{proof}
    \end{itemize}
\end{theo}

The safety lemma is one of the main reasons we use a three pahse protocol
instead of a two phase protocol.

\begin{itemize}
    \item If we only have two phases, we cannot gurantee a request has been commited,
    it will be prepared by a majority.
    \item Commited requests will not be recommited, this violates safety
\end{itemize}
\chapter{Consensus}
\section{Weak consensus}
\section{Agreement}
%\input{chap.tex}


\cite{Gao}

\bibliography{refs}

\end{document}